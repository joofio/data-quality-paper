% This is samplepaper.tex, a sample chapter demonstrating the
% LLNCS macro package for Springer Computer Science proceedings;
% Version 2.20 of 2017/10/04
%
\documentclass[]{article}
%
\usepackage{url}
\usepackage{footnote}
\usepackage{setspace}
\usepackage{graphicx}
\usepackage[british]{babel}
\usepackage[ruled,vlined]{algorithm2e}
\usepackage{lscape}
\usepackage{longtable}
\usepackage{tabularx}
\usepackage{booktabs}
\usepackage{xltabular}
\usepackage{array}
\usepackage{svg}
\usepackage{float} 
\usepackage{graphicx}
\graphicspath{ {./imgs/} }
\newcommand{\comment}[1]{}
\usepackage[affil-it]{authblk} % For author affiliations
\usepackage[ruled]{algorithm2e}
\usepackage[paperheight=10in,paperwidth=6.5in,margin=2cm,headsep=.5cm,top=2.5cm,headheight=1cm]{geometry}

\BeforeBeginEnvironment{appendices}{\clearpage}

\newcommand{\algorithmfootnote}[2][\footnotesize]{%
  \let\old@algocf@finish\@algocf@finish% Store algorithm finish macro
  \def\@algocf@finish{\old@algocf@finish% Update finish macro to insert "footnote"
    \leavevmode\rlap{\begin{minipage}{\linewidth}
    #1#2
    \end{minipage}}%
  }%
}

% Used for displaying a sample figure. If possible, figure files should
% be included in EPS format.
%
% If you use the hyperref package, please uncomment the following line
% to display URLs in blue roman font according to Springer's eBook style:
% \renewcommand\UrlFont{\color{blue}\rmfamily}
\doublespacing
%\begin{document}
%
\title{Development and Validation of a Data Quality Evaluation Tool in Obstetrics Real-World Data through HL7-FHIR interoperable Bayesian Networks and Expert Rules}
%
%\titlerunning{Abbreviated paper title}
% If the paper title is too long for the running head, you can set
% an abbreviated paper title here
%

\author[1,*]{João Coutinho-Almeida}
\author[2]{Carlos Saez}
\author[2]{Ricardo João Cruz-Correia}
\author[3]{Pedro Pereira Rodrigues}
\affil[1]{\small Department, Organization, Street, City, 100190, State, Country}
\affil[2]{Department, Organization, Street, City, 10587, State, Country}
\affil[3]{Department, Organization, Street, City, 610101, State, Country}
\affil[*]{Corresponding author: \texttt{user\_id@university.edu}}

\date{}  % Remove date

\begin{document}

\maketitle

\begin{abstract}
  \textbf{Background:} The increasing prevalence of Electronic Health Records (EHRs) in healthcare systems globally has underscored the importance of data quality for clinical decision-making and research, particularly in obstetrics. High-quality data is vital for an accurate representation of patient populations and to avoid erroneous healthcare decisions. However, existing studies have highlighted significant challenges in EHR data quality, necessitating innovative tools and methodologies for effective data quality assessment and improvement.

  \textbf{Objective:} This paper addresses the critical need for reliable data quality evaluation in obstetrics by developing a novel tool. The tool utilizes Health Level 7 (HL7) Fast Healthcare Interoperable Resources (FHIR) standards in conjunction with Bayesian Networks and expert rules, offering a comprehensive approach to assessing data quality in real-world obstetrics data.
  
  \textbf{Methods:} A harmonized framework focusing on Completeness, Plausibility, and Conformance underpins our methodology. We employed Bayesian networks for advanced probabilistic modeling, integrated outlier detection methods, and a rule-based system grounded in domain-specific knowledge. The development and validation of the tool were based on obstetrics data from nine Portuguese hospitals, spanning the years 2019 to 2020.
  
  \textbf{Results:} The developed tool demonstrated strong potential for identifying data quality issues in obstetrics EHRs. Bayesian networks used in the tool showed high performance for various features. Preliminary evaluations can reach AUROC of as high as 88\%. The tool's infrastructure and interoperable format as a FHIR API enables a possible deployment of a real-time data quality assessment in obstetrics settings. 
  
  \textbf{Conclusion:} This study contributes significantly to the field of EHR data quality assessment, with a specific focus on obstetrics. The combination of HL7-FHIR interoperability, machine learning techniques, and expert knowledge presents a robust, adaptable solution to the challenges of healthcare data quality. Future research should explore tailored data quality evaluations for different healthcare contexts, as well as further validation of the tool capabilities, enhancing the tool's utility across diverse medical domains.
\end{abstract}

\textbf{Keywords}: Data Quality, Machine-learning, FHIR, Real-world data, Bayesian Networks


%\author{\inst{1,2}\orcidID{0000-0003-0882-6547} \and  \inst{}\orcidID{} \and
%\inst{1,2}\orcidID{0000-0002-3764-5158} \and
%,\inst{1,2}\orcidID{0000-0001-7867-6682} }
%
%\authorrunning{J. Almeida et al.}

% Affiliations
%\affil{Affiliation One, Department, University}
%\affil{Affiliation Two, Department, University}


% First names are abbreviated in the running head.
% If there are more than two authors, 'et al.' is used.
%
%\institute{ CINTESIS@RISE - Centre for Health Technologies and Services Research, University of Porto, Portugal \and
%MEDCIDS – Faculty of Medicine of University of Porto, Portugal}
%
%\maketitle              % typeset the header of the contribution
%
%\begin{abstract}





%\keywords{Data Quality \and Machine-learning \and Bayesian Networks \and Real-world data \and FHIR}
%\end{abstract}
%
%Breast Cancer; personalised medicine; disease characteristics; machine learning; Cost-effectiveness; data warehousing; data integration
%
\section{Introduction}
\input{sections/intro}

\section{Background and Related Work}
\input{sections/relatedwork}

\section{Materials}
\input{sections/materials}

\section{Methods}
We wrote all of the code in Python 3.10.6 with the usage of the scikit-learn library for preprocessing, and evaluation \cite{scikit-learn}. 
For plausibility, a Bayesian network was used. We used this model due to the possiblity of using a single model for classifying the plausibility of all columns and due to its interpretable nature.
The networks was created with the pgmpy package \cite{pgmpy}. For creating the network, all null representations were standardized. Data was prepossessed with the removal of features with high missing rates ($>$ 80\% overall). All missing value representations were standardized. The imputation process was performed with the median for continuous and a new category (NULLIMP) for categorical variables. Then, the continuous variables were discretized into three bins defined by quantile. The evaluation was done with cross-validation with 10 splits and two repetitions for each column as the target. 

As for  Z-Scores, they were defined for all continuous variables based on the interquartile range. Then, rows were also assessed with distance analysis, with Local Outlier Factor and Elliptic Envelope from scikit-learn and the outlier-tree algorithm. We also added a rule engine, using the \textit{great\_expectations} package. Rules were defined by the team, focusing on impossible numbers present in age, weight, or relationship between variables. As for missing information was created with all the data, creating the scoring based on the inverse of the missing percentage. Missing detection was based on primary key variables. For completeness, we used the inverse of the percentage of nulls in the training set.
The API for serving the prediction models was developed with FastAPI. So, the methods applied in terms of the DQA framework shown in figure \ref{fig:categories} are described in the table \ref{tab:methods}.

\begin{table}[htpb]
\caption{Implemented Methods} \label{tab:methods}
\renewcommand{\arraystretch}{1.4}
\setlength{\tabcolsep}{10pt}

\begin{tabularx}{\textwidth}{ p{2cm} p{3.5cm} X }
\hline
 Category   & Subcategory           & Method   \\ \hline
Completeness     & N/A               & Score by the inverse percentage of missing in the train data         \\ 
Plausibility & Atemporal Plausibility & Bayesian model prediction based on the other values of row \\ 
Plausibility & Atemporal Plausibility         & Z-score for column value based on IQR train data       \\    
Plausibility & Atemporal Plausibility           & Elliptic Envelope                       \\ 
Plausibility & Atemporal Plausibility           & Local Outlier Factor                \\ 
Conformance & Value Conformance           & Manual Rule engine                           \\ 
Plausibility & Atemporal Plausibility           & Manual Rule engine                      \\ 
Plausibility & Atemporal Plausibility           & outlier-tree                      \\ 
\hline
\end{tabularx}

\end{table}
The method of scoring was to obtain a single value that could grasp the quality of the row or patient.
To assess the tool's usefulness, we will implement it in a production environment and collect metrics regarding the data being produced. Then we intended to present some results to selected obstetrics clinicians for them to assess how likely the information is to be suitable for usage. We will also compare the results with the ones from the model to make sanity checks regarding the model's performance and adequacy. The metrics of agreements will be the Cohen kappa and a ranking metric -  Normalized Discounted Cumulative Gain (NDCG) \cite{wangTheoreticalAnalysisNDCG} which is the sum of the true scores ranked in the order induced by the predicted scores, after applying a logarithmic discount. Then divide by the best possible score to obtain a score between 0 and 1.



\section{Results}
\hl{Our main result is the tool we developed that we are going to further explore its components. The main one is the Bayesian network developed, and its structure is presented in figure} \ref{fig:network}. \hl{The example in the image shows the probability for all the classes in the category of Robson group, taking into account the values of the other categories (both known and unknown). In this case, the probability for Robson group number 3 is 77.41\%. }
%TC:ignore
\begin{figure}[h!]
    \centering
    \caption{\hl{Bayesian Network learned. Nodes acronyms are explained in appendix 1. The example shows the inference for the Robson Group (10 categories) and the probability of each category, given a set of other features.} }\label{fig:network} 
    \includegraphics[scale=0.30]{imgs/new-bn.png}
    \end{figure}
    %TC:endignore


\hl{The results of the cross validation can be seen in the table} \ref{tab:result_auc}.\hl{The average Area Under the Receiver Operating Characteristic Curve was 0.857. In parentheses is the number of non null rows that were used in the validation for that column as target.}



    
\begin{table}[h!]
    \caption{Repeated Cross-Validation (10x2) Results: Column description with AUROC along with 95\% CI. (n) is the number of non null rows.} \label{tab:result_auc} 
   
   \renewcommand{\arraystretch}{1.2}
   %\setlength{\tabcolsep}{8pt}
   \centering
   \begin{tabular} {lcc }
    \toprule
    Name of Variable (n) &               Average &               95\% CI \\
 \midrule
   Nr of previously born babies (44387) & 0.944 & [0.943, 0.945] \\
   Nr pregancies (73335) & 0.797 & [0.778, 0.816] \\
   Nr eutotic deliveries (28809) & 0.969 & [0.968, 0.969] \\
   Nr Prev. C-section (17879)& 0.958 & [0.958, 0.958] \\
   Mother’s Age (73337) & 0.638 & [0.637, 0.638] \\
   Mother’s weight start (63324)& 0.881 & [0.88, 0.881] \\
   BMI (62260) & 0.881 & [0.881, 0.882] \\
   Nr Prenatal Consultations (61388) & 0.75 & [0.75, 0.75] \\
   Nr Weeks on admission (72715) & 0.968 & [0.968, 0.969] \\
   Pregnancy weeks on delivery (73217) & 0.974 & [0.974, 0.974] \\
   Nr deliveries with vacuum (15985) & 0.974 & [0.974, 0.974] \\
   Pregnancy Type (64517) & 0.728 & [0.726, 0.73]\\
   If pregnancy was accompanied in the hospital (49738)& 0.894 & [0.893, 0.895] \\
    If delivery was spontaneous (26360) & 0.816 & [0.815, 0.816] \\
    Baby’s position admission (20166)& 0.751 & [0.743, 0.758] \\
    Robson Group (69280) & 0.931 & [0.93, 0.932] \\
    If pregnancy was accompanied (73219) & 0.983 & [0.982, 0.983] \\
    Delivery Type (73350)& 0.866 & [0.865, 0.868] \\
    If was accompanied in the primary care setting (49812) & 0.79 & [0.789, 0.791] \\
    Baby’s position delivery (73227) & 0.942 & [0.938, 0.946] \\
    Blood Group (73132) & 0.514 & [0.507, 0.52] \\
    Hospital ID (73352) & 0.896 & [0.896, 0.897] \\
    If accompanied in a private care setting (18049) & 0.771 & [0.77, 0.772] \\
    Actual Type of delivery (65606) & 0.952 & [0.951, 0.952] \\
   \hline
    \multicolumn{3}{c}{\textbf{Average}  \textbf{0.857 [0.846, 0.868]}} \\
   
   \hline
   \end{tabular}
   \end{table}





\subsection{Deployment \& Preliminary Validation}

The purpose of this model is to serve as an API for usage within a healthcare institution and act as a supplementary data quality assessment tool. Although a concrete, vendor-specific information model and health information system were initially used, our goal is to develop a more universal clinical decision support system. This system should be usable across all systems involved in birth and obstetrics departments. Therefore, we constructed it using the Health Level 7 (HL7) Fast Healthcare Interoperable Resources (FHIR) R5 version standard. This approach simplifies the process of API interaction. Rather than utilizing a proprietary model for the data, we based our decision on the use of FHIR resources: Bundle and Observation. These resources handle the request and response through a customized operation named "\$quality\_check". We intend to publish the profiles of these objects to streamline API access via standardized mechanisms and data models. The model then makes use of the customized operation and of several base resources to construct a FHIR message, which are: Bundle, MessageHeader, Observation, Device. Observation is where the information about the record is contained, Device contains information about the model, and MessageHeader is used to add information about the request. Finally, the Bundle is used to group all of these resources together. The current version of the profiles can be accessed here\unskip~\cite{obs-ig}. 

We conducted a preliminary validation of the tool to assess its initial performance and gather early insights, although a formal, comprehensive assessment was not performed at this stage. In order to do so, we deployed the tool in docker format in a hospital to gather new data. We gathered 3223 new cases and returned a score for quality as exemplified in figure \ref{fig:scores}. Being that the score is from 0 to 1, the average score was 0.75 and IQR was 0.016. The formula gives weights to different dimensions since we feel some are more robust than others. We gave more weight to rule system, and gave less to the missing and IQR score. Another component of this initial validation was to gather clinicians evaluation of random data points from the real world deployment and compare them with the tool's assessment. We got 4 answers. Figure \ref{fig:clinical} shows the distribution of the perceived quality of each record.


%TC:ignore
\begin{figure}[htbp]
\centering
\caption{Model score for newly seen data}\label{fig:scores} 
\includegraphics[scale=0.78]{imgs/Scoring_V2.png}
\end{figure}
%TC:endignore

%TC:ignore
\begin{figure}[htbp]
\centering
\caption{Distribution of rankings obtained from the assessment of 10 records by 4 different clinicians. Y is the distribution of clinicians' assessment, X is the patient ID.}\label{fig:clinical} 
\includegraphics[scale=0.52]{imgs/clinical_assessment_no_model.png}
\end{figure}
%TC:endignore

%The Average Spearman's Rank Correlation Coefficient was 0.14 and the Kendall's Tau was 0.096 with a \textit{p-value} of 0.712.
Figure \ref{fig:auc_changes} shows the performance of the model with several ranking thresholds to differentiate bad quality record from good quality record. Each line/color is a threshold (3,4,5,6) and the AUC is shown in the label. The Average Spearman's Rank Correlation Coefficient was 0.42 (p-value: 0.23) and the Kendall's Tau was 0.3 (p-value: 0.2). Both tests were based on a $\alpha$ of 0.05.

%TC:ignore
\begin{figure}[htbp]
    \centering
    \caption{Model Performance in terms of AUROC, depending on the threshold defined on the physician assessed data. The colors show different threshold used to consider a bad quality record given the average ranking. Label shows the threshold and respective AUROC.}\label{fig:auc_changes} 
    \includegraphics[scale=0.78]{imgs/auroc_curve_threshold.png}
    \end{figure}
    %TC:endignore

\section{Discussion}
The first thing to address is that data quality is still an elusive concept since it has a contextual dimension and the quality of the record depends on the usage of the information. For example, data aimed at primary usage and day-to-day healthcare decisions about a patient will have different requirements regarding the importance of some variable or completeness of information very different from data needed to create summary statistics for key performance indicators extraction. 
Moreover, the data is still very vendor-specific. Even though we used an interoperability standard, the semantic layer, more connected with terminology is still lacking. This is an issue to be addressed in order to improve the interoperability of the standard. Moreover, we do not know how the training done with this data is generalizable to other vendors. One opportunity arises of mapping all of this data to a widely used terminology like snomed or loinc. Nevertheless, the usage of FHIR and the fact that the data is mapped to a standard terminology, makes it easier to use the data in other systems and to compare the results with other studies. Furthermore, being available freely and online makes it easier to understand how to map vendor specific datasets to the model and use it in other contexts.


%discutir que dependendendo da pessoa /context/profissao/objecti pode ser preciso dar pesos diferentes a diferentes colunas.
%para dar GDH nao ha coisas que sao precisas mas ha coisas vitais
%para financeiro ha so umas precisas (tipo meds e procs)
%para clinico ha outras e por ai fora

Regarding the clinical evaluation, our results showed great variability between clinicians and between clinicians and the model. This was already expected since data quality is a quite evasive concept and highly contextual and subjective evaluation. However, we do see a tendency for a better assessment of the worst quality results than the best ones. This is also related to comments we received from evaluators, where they could benefit from marking records of equal quality in their perspective, and being unable to do so, they would rely on guts or intuition to make the decision. This is a very important aspect to take into account when designing the evaluation of data quality.
This result suggests that the system may not be suitable to classify and rank good-quality records but could be useful to alert for low-quality ones, which is also a very important task with a great impact on the quality of the data. These findings are supported not only by \ref*{fig:clinical} but also by figure \ref*{fig:scores} where we can add some threshold for the need for a human review. From the preliminary data in the questionnaires and looking at the graph, we believe that a threshold of around 0.3 could be a good starting point. This is a very important aspect to take into account when designing the evaluation of data quality.
Aligned with this is the fact that the system relies on existing methods of trying to explain the data, and outlier-tree and the bayesian network are vital to that task. If explainability and interpretability are important, this need only increases when we are dealing with such subjective concepts as data quality.
However, it is not perfect and it is not always able to explain the data. This is a limitation of the system and it is important to address it in future work.

\section{Conclusion}
This work is still an early draft of a production-ready tool. However, we feel the work done is already a valuable insight into how to use data quality frameworks and several statistical tools in order to assess ehr data quality.
This is a fundamental process not only to guarantee the quality of data for primary usage on a day-to-day but also for securing quality for secondary analysis and usage.
For the next steps, we would like to further evaluate the score and its relationship with clinical usefulness. This would also include a further assessment of a  threshold for the score for defining a record that would require human attention.

%
% ---- Bibliography ----
%
% BibTeX users should specify bibliography style 'splncs04'.
% References will then be sorted and formatted in the correct style.
%
% \bibliographystyle{splncs04}
% \bibliography{mybibliography}
%
%\begin{thebibliography}{8}
%\bibliographystyle{splncs04}
\bibliographystyle{unsrt}

\bibliography{bibliography}
%\end{thebibliography}
\appendix

\section{Data Dictionary}
\label{appendix:data_dict}
\begin{table}[H]
\begin{tabularx}{\textwidth}{| l  | X |}
\toprule
Initial  &    Description \\
\midrule
IA &  Mother Age \\
GS  & Blood Group \\
PI &   Weight at the beginning of pregnancy \\
PAI &  Weight on Admission \\
IMC &   BMI \\
CIG &   If Smoker During Pregnancy \\
APARA  & Number of previously born babies\\
AGESTA  &   Number of Pregnancies   \\
EA &     Number of Previous Eutocic Deliveries with no assistance \\
VA &      Number of Previous Eutocic Deliveries with help of vacuum extraction \\
FA &     Number of Previous Eutocic Deliveries with help of forceps \\
CA &   Number of  Previous C-sections \\
TG &     Pregnancy Type (spontaneous, In vitro fertilisation...) \\
V &     If the pregnancy was accompanied by physician \\
NRCPN &     Number of prenatal consultations \\
VH &      If the pregnancy was accompanied by a physician in a hospital \\
VP &    If the pregnancy was accompanied by a physician in a private clinic \\
VCS &  If the pregnancy was accompanied by a physician in a primary care facility \\
VNH &    If the pregnancy was accompanied by a physician in the hospital the delivery was made \\
B  & Pelvis Adequacy \\
AA & Baby's Position on Admission \\
BS &  Bishop Score \\
BC &   Bishop Score Cervical Consistency\\
BDE &  Bishop Score Fetal Station \\
BDI &  Bishop Score Dilatation \\
BE  &   Bishop Score Effacement \\
BP & Bishop Score Cervical Position \\
IGA  & Number of Weeks on Admission \\
TPEE  & If the delivery was spontaneous   \\
TPEI  &  If the delivery was induced  \\
RPM  & If there was a rupture of the amniotic pocket before delivery began \\
DG &  Gestational Diabetes \\
TP & Delivery Type \\
ANP  & Baby's Position on Delivery \\
TPNP  & Actual Type of Delivery\\
SGP  & Pregnancy Weeks on Delivery \\
GR  &   Robson Group \\
\bottomrule
\end{tabularx}
\end{table}




\end{document}
