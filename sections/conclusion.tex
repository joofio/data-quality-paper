We believe the work done is already a valuable insight into how to use data quality frameworks and several statistical tools in order to assess EHR data quality in real time. This is a fundamental process not only to guarantee the quality of data for primary usage but also for securing quality for secondary analysis and usage. We believe the fact that we created an interoperable tool that was trained on real obstetrics data from 9 different hospitals and has the ability to provide a single score for a clinical record can help institutions, academics, and EHR vendors implement data quality assessment tools in their own systems and institutions. With the further evaluation of the score and its relationship with clinical usefulness and a further assessment of a threshold for the score for defining a record that would require human attention would be vital to apply this tool in production with high levels of trust and quality.