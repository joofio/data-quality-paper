With the wide spreading of healthcare information systems across all contexts of healthcare practice, the production of health-related data has followed this incremental behaviour. The potential for using this data to create new clinical knowledge and push medicine further is tempting \cite{martin-sanchezBigDataMedicine2014}.
\hl{However, to correctly use the data stored in Electronic Health Records (EHRs), the quality of the data must be robust enough to sustain the clinical decisions made based on this data. Data quality cannot be understood as a straightforward concept; it is highly dependent on the context in which it is evaluated. The quality thresholds and dimensions required to classify the quality of the data depend on the purpose that we intend to use that very same data} \cite{waljiElectronicHealthRecords2019}. These uses can be very distinct and have different impacts as well. For one, we can use data to support day-to-day decisions regarding individual patients' care \cite{verheijPossibleSourcesBias2018}. These decisions can include ones based on recorded information to understand a patient's history, clinical decision support systems based on this data, or even using the data to help support a more macro, public health-oriented decision. Another area is using information for management purposes. The data can be used by management bodies and regulatory authorities to extract metrics regarding the quality of care or reimbursement purposes. Thirdly, data can be used for research purposes, namely observational studies and, more recently, to support clinical trials through real-world evidence analysis \cite{coreyAssessingQualitySurgical2020,verheijPossibleSourcesBias2018,wengClinicalDataQuality2020}. 
So, all the EHR data-based decisions can only be as good as the data supporting them. Several studies have already warned about the lack of data quality in EHRs and how this can be a significant hurdle to an accurate representation of the population and potentially lead to erroneous healthcare decisions \cite{reimerDataQualityAssessment2016a,joukesImpactElectronicPaperBased2019a,huserMultisiteEvaluationData2016,zhangUnderstandingDetectingDefects2020,kramerImpactDataQuality2021,gigantiImpactDataQuality2019}.

There are several steps in the data lifecycle that can be prone to error, from data generation, where the data is registered by healthcare professionals, passing by data processing, whether inside healthcare institutions or by software engineers aiming to reuse data, to data interpretation and reuse, where investigators try to interpret the meaning of registered data \cite{wengClinicalDataQuality2020}.
So, with all the data's possible uses added to the several steps that can introduce errors throughout the data lifecycle, data quality frameworks and sequential implementations can have very distinct approaches and methodologies to assess data quality. Data quality tools for checking data being registered live to support day-to-day decisions will be significantly different from one whose only purpose is to provide quality checks for research purposes. So, methodologies to tackle these issues are necessary for guaranteeing the quality of healthcare practice and the knowledge derived from EHR data. 

%In this paper, we aim to achieve the following objectives:
%\begin{itemize}

%\item Identify and Explain Potential Issues in Full Deployment: We aim to enlighten readers on the various challenges and issues that may arise when fully deploying a tool designed for improving data quality in obstetrics. This involves a detailed analysis of potential technical, operational, and ethical concerns.

%\item Develop a Single Data Quality Score: We propose the creation of a comprehensive single score for data quality. This score will facilitate the comparison of high-quality and low-quality records within a database, enabling a more standardized and efficient assessment of data quality.

%\item Evaluate Tool Performance in Early-Stage Real-World Scenarios: Our objective is to assess how the proposed tool functions in early-stage real-world scenarios. This includes examining its effectiveness in collaboration with obstetricians and identifying practical strategies for improving data quality based on real-world feedback and conditions.
%\end{itemize}

%Consequently, in this paper, we propose:
%\begin{itemize}
%    \item Enlighten on the issues that can appear with a full deployment of such a tool;
%    \item Suggestion of a creation of a single score for data quality for comparison of high-quality and low-quality records in a database.
%    \item Assess how such a tool can work in early-stage real-world scenarios and how to work with obstetricians to improve data quality.
%    \item Identify data quality issues on obstetrics data
%\end{itemize}




