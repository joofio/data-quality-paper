This work adds several pieces of information to the state of the art of data quality analysis. First we tried to map the output of an automatic assessment tool to the human perception of quality and the issues linked to doing so. Secondly, the fact that we applied explainable machine learning methods such as bayesian networks to leverage the potency of advanced data analysis without compromising interpretability and explainability. Furthermore, a single model was able to reach high performance metrics for almost all variables. Thirdly, the fact that interoperability standard such as FHIR can be adopted to facilitate the usage and information exchange of such tools. However, there are also shortcoming and challenges to address. The first is that data quality is still an elusive concept since it has a contextual dimension and the quality of the record depends on the usage of the information. For example, data aimed at primary usage and day-to-day healthcare decisions about a patient will have different requirements regarding the importance of some variable or completeness of information very different from data needed to create summary statistics for key performance indicators extraction. Moreover, the data is still very vendor-specific. Even though we used an interoperability standard, the semantic layer, more connected with terminology is still lacking. This is an issue to be addressed in order to improve the interoperability of the standard. Moreover, we do not know how the training done with this data is generalizable to other vendors. One opportunity arises of mapping all of this data to a widely used terminology like SNOMED CT or LOINC. Nevertheless, the usage of FHIR and the fact that the data is mapped to a standard terminology, makes it easier to use the data in other systems and to compare the results with other studies. Furthermore, being available freely and online makes it easier to understand how to map vendor-specific datasets to the model and use it in other contexts. Regarding the model, the usage of explainable methodologies like outlier-tree and transparent models like Bayesian networks are vital for clinical application. Since we use a single model to classify possible errors in the records, the ability to try to show clinicians why that value was tagged is of uttermost importance in order to get feedback and action from humans. From the experience gathered with the study, we believe that a weaker but transparent model could have more impact than better performant but opaque ones. If explainability and interpretability are important for any ML problem, this need only increases when we are dealing with such subjective concepts as data quality.

\hl{In terms of domain-specific issues, particularly in obstetrics, we found that assessing the quality of a record in an EHR is not an easy task for clinicians. We discovered that for a proper assessment, a context and objective must be defined in order to make the evaluation more objective and manageable. Moreover, the ranking methodology, though very useful for comparison with the model, presents challenges for clinicians who find it difficult to order 10 records when some appear to be of equal quality.} This is a very important aspect to consider when designing an evaluation method for data quality. Perhaps a categorical evaluation of yes/no would be more effective than ordering several records. These reasons might explain the great variability between clinicians (figure \ref{fig:clinical}) and between clinicians and the model (Spearman and Kendall tau). Despite that, our preliminary results are promising, demonstrating an AUROC curve for categorizing bad quality records as high as 88\% and low as 56\%. The highest value was achieved by classifying all record with a mean rank of 4 or above as bad quality and the others as good quality records. However, these results rely on very few samples, so more data and research are needed in this area since it is a very subjective decision, and it should take into account the context and the objective of the evaluation. For example, if the objective is research use, the weights given to each dimension can be a set. On the other hand, if the objective is to use the data for day-to-day clinical decisions, another set of weights could be used. 

For the next steps, a promising research direction would be identifying contexts for applying data quality checks like primary usage, research purposes, and aggregated analysis for decision-making among others. This could enhance targeting those contexts and understanding the importance of each variable for those use cases. Incorporating this approach into the tool to weigh the different variables according to the context would be beneficial.  Finally, gaining access to more data and clinician evaluation of records, although challenging, is important to thoroughly assess the performance of the tool.

