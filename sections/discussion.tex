The first thing to address is that data quality is still an elusive concept since it has a contextual dimension and the quality of the record depends on the usage of the information. For example, data aimed at primary usage and day-to-day healthcare decisions about a patient will have different requirements regarding the importance of some variable or completeness of information very different from data needed to create summary statistics for key performance indicators extraction. 
Moreover, the data is still very vendor-specific. Even though we used a interoperability standards, the semantic layer, more connected with terminology is still lacking. This is a issue to be addressed in order to improve the interoperability of the standard. Moreover, we do not know how the training done with this data is generalizable to other vendors. One opportunity arises of mapping all of this data to a widely used terminology like snomed or loinc.

%discutir que dependendendo da pessoa /context/profissao/objecti pode ser preciso dar pesos diferentes a diferentes colunas.
%para dar GDH nao ha coisas que sao precisas mas ha coisas vitais
%para financeiro ha so umas precisas (tipo meds e procs)
%para clinico ha outras e por ai fora